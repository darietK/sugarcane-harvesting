\documentclass[10pt,letterpaper]{article}
\begin{document}
\title{PROBLEM OF MISS CALCULATION OF HARVESTED SUGAR CANE TONES AND SOLUTIONS TO BE TAKEN BY OUT GROWERS TO SOLVE THE PROBLEM IN UGANDA.}
\author{by KAMUKAMA DARIET  \\ 216012298 \\  16/U/5346/PS}
\maketitle
\section{Introduction }
Sugar cane out growers in Uganda have got a problem of not measuring the right amount of tones harvested from their plantations by the sugar cane processing companies where less tones are measured more than the quantity of tones harvested. 
It causes a less pay from the cane processing companies to the out growers since less tones are measured than the harvested ones.
\section{Problem statement }
The purpose of this study is to identify the problem of less measurement of tones and the use of computer aided machines to measure right amount of tones harvested.
\section{Scope of the study}
This study was limited to only Sugar cane out growers in Uganda where the effects and the ways on how to reduce miss calculation of sugar cane tones harvested were discussed.
\section{Significance of the study}
To determine useful information to help calculate sugar cane tones harvested and how the computer aided machines will help solve the problem.
\section{Main objective of the study}
To examine the effects and characteristics of less measurement of harvested sugar cane tones and the common method used in measuring the harvested sugar cane tones.
\section{Specific objectives}
To know the different ways of how sugar cane harvesting is done and alert out growers to be careful and always be present while the harvesting takes place.
To emphasize the use of computerised harvesters that will reduce the miss calculation of harvested sugar cane tones.
To determine whether sugar processing companies will provide computerised machine harvesters during harvesting of the sugar canes. 
\section{Research methodology}
The data was obtained using analytical research where from the internet I managed to obtain some facts about miss calculation of harvested sugar cane tones and I was able to analyse its effects onto the out growers and the different ways to reduce the problem.
Quantitative methodology was used where I sampled a group of out growers in Lugazi and I found out that more than one hundred out growers have been affected by miss calculation of harvested sugar cane tones and paid less.
\section{Recommendations}
People should always keep alert of when their sugar cane plantations are to be harvested. This creates awareness and makes the person present while harvesting.
Out growers should contact police in case of any act of harvesting a plantation without a computerized machine harvester.


\end{document}